\hypertarget{index_intro_sec}{}\section{Hash\+Set}\label{index_intro_sec}
\hyperlink{class_hash_set}{Hash\+Set} to kolekcja unikatowych elementow ktora wykorzystuje funkcje \textquotesingle{}hashujaca\textquotesingle{} Rozwiazaniem jest tablica list jednokierunkowych. Numer komorki do ktorej trafi element jest obliczany za pomoca funkcji \textquotesingle{}hashujacej\textquotesingle{}. Jesli napotkany element juz istnieje nie jest on dodawany. Na samym poczatku wszystkie komorki sa puste. Uzycie szablonu pozwala na uzywanie elementow roznego typu(int, string, float itp.). \hypertarget{index_intro_sec3}{}\section{Metody i operatory}\label{index_intro_sec3}
Opisane jest to w zakladce \hyperlink{class_hash_set}{Hash\+Set} $\ast$\hypertarget{index_intro_sec4}{}\section{Mieszkanie}\label{index_intro_sec4}
\hypertarget{index_step1}{}\subsection{Przeglad}\label{index_step1}
\hyperlink{class_mieszkanie}{Mieszkanie} jest klasa reprezentujaca mieszkanie. Pola jakie zawiera klasa \+: \begin{DoxyItemize}
\item string {\bfseries nazw\+Wlasciciela} -\/ nazwa wlasciciela mieszkania\item float {\bfseries powierzchnia} -\/ powierzchnia mieszkania\item int {\bfseries liczba\+Pokoi} -\/ liczba pokoi w mieszkaniu\item bool {\bfseries zamieszkane} -\/ mowi o tym czy mieszkanie jest zamieszkane czy puste \item static unsigned {\bfseries liczba\+Wczytanych\+Mieszk} -\/ licznik poprawnie wczytanych obiektów mieszkań ze strumieni \end{DoxyItemize}
\hypertarget{index_intro_sec10}{}\section{Metody i operatory}\label{index_intro_sec10}
Opisane jest to w zakladce \hyperlink{class_mieszkanie}{Mieszkanie}.\hypertarget{index_intro_sec5}{}\section{Bucket}\label{index_intro_sec5}
\hypertarget{index_step1}{}\subsection{Przeglad}\label{index_step1}
\hyperlink{struct_bucket}{Bucket} jest struktura ktora umozliwia tworzenie listy jednokierunkowej do ktorej dodawana sa elementy w odpowiednich komorkach tablicy. \hypertarget{index_intro_sec10}{}\section{Metody i operatory}\label{index_intro_sec10}
Opisane jest to w zakladce \hyperlink{struct_bucket}{Bucket}. \hypertarget{index_intro_sec9}{}\section{Co zostalo dodane}\label{index_intro_sec9}
Do projektu 2 dodałem dwie klasy pochodne\+: \begin{DoxyItemize}
\item finalna \hyperlink{class_mieszkanie_hipoteczne}{Mieszkanie\+Hipoteczne}\item \hyperlink{class_mieszkanie_wynajmowane}{Mieszkanie\+Wynajmowane}\item a także wirtualna klase bazowa \hyperlink{class_siedziba}{Siedziba}\end{DoxyItemize}
W mainie pokazalem przykladowe rzutowanie w dol i w gore, wskazniki na obiekty. Oczywiscie zgodnie z wymogiem uzylem dodatkowo slowa kluczowego protected w klasie \hyperlink{class_siedziba}{Siedziba}.\hypertarget{index_intro_sec6}{}\section{Obsluga sytuacji wyjatkowych\+:}\label{index_intro_sec6}
Została zastosowana przy uzyciu funkcji remove(). Przy probie usuniecia nieistniejacego obiektu, rzucany jest obiekt typu T.\hypertarget{index_intro_sec7}{}\section{Iterator\+:}\label{index_intro_sec7}
Zostala utworzona klasa iteratora, ktora zostala zaimplementowana z pliku z Hash\+Setem. Zawiera funkce zgodne z zalozeniami projektu(next, prev, begin, end...) \hypertarget{index_intro_sec8}{}\section{Operatory strumieniowe w klasach\+:}\label{index_intro_sec8}
W klasie \hyperlink{class_siedziba}{Siedziba} zostal przeciazony operator $<$$<$ W klasie \hyperlink{class_mieszkanie_wynajmowane}{Mieszkanie\+Wynajmowane} zostaly przeciazone operatory $<$$<$,$>$$>$ W klasie \hyperlink{class_mieszkanie_hipoteczne}{Mieszkanie\+Hipoteczne} nie przeciazylem zadnych operatorow w celu pokazania efektu 